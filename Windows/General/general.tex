documentclass{article}
\usepackage{graphicx}
\usepackage{hyperref}
\usepackage{listings}
\usepackage{xcolor}
\usepackage{tikzsymbols}
\usepackage{float}

\lstset{
    basicstyle=\ttfamily,
    backgroundcolor=\color{gray!30},
}

\title{Overview}
\begin{document}
\maketitle

\graphicspath{ {./Images/} }
\tableofcontents

\section{Introduction}
I was dissatisfied with most of what I have done. It is maybe passable as bad technical documentation, but it does not get at the heart of the operating system,
the \textit{why} of windows. So here I will explain things at a more conceptual level. Hopefully it is helpful.

\section{Firewalls}
\subsection{TCP View}

\subsection{netstat}
netstat -anob

-a: Displays all active network connections and listening ports.

-n: Displays addresses and port numbers in numerical form (rather than resolving names).

-o: Displays the owning process ID associated with each connection.

-b: Displays the executable involved in creating the connection or listening port. Note: Using -b can be slow and might require elevated privileges.

\section{Active Directory}
\subsection{Initial Account}
The domain controller administrator account is the same account as the local administrator account. So if you disable the local administrator on the DC,
that will also disable the domain administrator. So don't do that!

\subsection{Disabling does not log out}
Merely disabling a user does not log them out. You also need to logoff the user. I have a script to do it.

\subsection{Kerberos / Kerberoasting}

\subsection{Service Accounts}

\subsection{Groups}

\subsection{Attacking DSRM password}

\subsection{Adding Linux to AD}

\section{Certificates}
\subsection{What is a certificate, what are we signing}
\subsection{Setting up auto enrollement}
\subsection{Client vs Server Certificates}

\subsection{Web Enrollement}
Chain of trust

http://<YourCA_Server_Name>/certsrv

\subsection{Certificate Templates}

\section{Processes}

\section{Services}
\subsection{Task Scheduler}

\section{DLLs}

\section{IIS}
ApplicationPoolIdentity: This is a unique identity for each application pool. It runs under a dynamically created, less privileged account. Starting from IIS 7.5, this became the default and is generally recommended because it provides a higher level of isolation between different web applications.

If you select "Use the built-in application pool identity", it will use the ApplicationPoolIdentity, which is the default and a safer option for running your web applications.

\section{DNS}
\subsection{Default Entries}
\subsection{Domain Transfers}
\subsection{DNSSEC}
\subsection{Reverse Lookups (bad)}
\subsection{Types of records}

\section{Registry}
\subsection{Autostart Locations}
\subsection{Password Filters}
When a user attempts to change their password or an administrator tries to reset it, the system will call functions in all registered password filter DLLs to validate the new password.
Registry Check: Examine the aforementioned registry key (HKEY_LOCAL_MACHINE\SYSTEM\CurrentControlSet\Control\Lsa\Notification Packages) to see if any non-default entries are present. Default values typically include entries like "scecli" but any additional values could indicate the presence of a custom password filter.

\section{Drivers}