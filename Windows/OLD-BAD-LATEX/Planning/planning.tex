\documentclass{article}
\usepackage{graphicx}
\usepackage{hyperref}
\usepackage{listings}
\usepackage{xcolor}
\usepackage{tikzsymbols}
\usepackage{float}
\usepackage{epigraph}

\lstset{
    basicstyle=\ttfamily,
    backgroundcolor=\color{gray!30},
}

\renewcommand{\epigraphflush}{flushleft}
\title{Planning}
\begin{document}
\maketitle

\graphicspath{ {./Images/} }
\tableofcontents

\section{Introduction}
This document is dedicated to how to learn the material, what in the 
documents needs improving, and how to prepare for the competition.

Ironically, this section needs to be fleshed out more. 
It will be as time goes on. I think the most valuable thing is to edit it after your experience with the competition.
Then we can figure out what we did wrong and what we need to improve on.

\section{Practice}
\epigraph{Theory and practice are the same in theory, but not in practice}{Ben Finegold}

\subsection{Take notes from the red team}
\epigraph{If you want to win, you must understand why you have been losing}{Internet person}

\section{Incident Response Templates}
We lost a lot of points due to a failure to do incident response. You should make a report for literally everything you do.
They give a brief tempalte to use, you should use that.

My year we submitted one giant incident response and it was terrible. Make a lot of smaller ones, and submit them quickly.

\section{Injects}
\epigraph{Dance like no one is watching; email like it may one day be read aloud in a deposition.}{Olivia Nuzzi}

They do not like it if you joke in your response, even if they joke in their inject.

\section{First Contact}
\epigraph{I am speed}{Lightning McQueen}

Speed is very important. I am not sure of how long you have before the red team attacks you, but it is not more than 10-15 minutes.
By this time, you should have removed all low hanging fruit so they do not mess you up before you can even begin.

This is my tentative starting plan for the 2024 competition:
\begin{enumerate}
        \item Run initial scripts (~5 min completion?)
        \item Look around with Process Explorer
\end{enumerate}

\subsection{Initial Scripts}
Your scripts should automate the menial work for you. This includes:
\begin{enumerate}
        \item Clearing the autoruns, clearing the task scheduler, correcting the registry (and writing the changes down)
        \item Setting up logs / confirming logs are set up.
        \item Downloading antivirus programs
        \item Disabling AD accounts (not the black team!)
        \item Disabling all local accounts
        \item Changing your password (and obfuscating it to negate keyloggers)
        \item Booting all RDP \& SSH users off of the machine (important, disabling accounts is not enough)
        \item Removing GPO policies
        \item Configuring the local firewall
        \item Run Antivirueses (Norton Power Eraser \& Kaspersky)
\end{enumerate}

\end{document}