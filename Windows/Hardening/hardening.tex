\documentclass{article}
\usepackage{graphicx}
\usepackage{hyperref}
\usepackage{listings}
\usepackage{xcolor}
\usepackage{tikzsymbols}
\usepackage{float}
\usepackage{epigraph}

\lstset{
    basicstyle=\ttfamily,
    backgroundcolor=\color{gray!30},
}

\renewcommand{\epigraphflush}{flushleft}

\begin{document}

\graphicspath{ {./Images/} }
\tableofcontents

\section{Introduction}
\epigraph{I remembered how Albert Camus talks about the concept of resistance. 
The idea is that if you see that you cannot win, do everything in your power to resist. And that memory gave me the determination I needed.}
{Ding Liren}

Assume everything is compromised. Even the laptops you are using for the competition! Be paranoid!

Also assume that the red team has fully gone through your public material the machines from previous years, as
I know they have gloated of busting teams who reuse passwords across multiple years.
They have a wordlist of all of the passwords used in past competions and run through it. 
On the topic of passwords, one red teamer gloated that he could enumarate all 12 character passwords in some short amount of time (6 hours?).
The red team does have the entire night to crack hashes. So make long passwords!

This is probably especially relevant for windows, as kerberos uses hashes extensively. <update with more details>

\section{Active Directory Freebies}
There are some very easy and quick ways to harden active directory.

\begin{enumerate}
        \item Disable all of the accounts besides the black team and your own
        \item Change your password to something random and good
        \item At some point, you can make a new administrator account with an unusual username. You can keep the old account around as a honeypot if you want
        \item After disabling accounts, you must kick them off of the machine! Merely disabling the account does not kick the user off of the machine
        \item You must also view the ssh connections and kick them off. (both ssh and rdp)
        \item Look at GPO and disable any malicious policies, such as policies that disable microsoft defender.
\end{enumerate}

\section{Local Account Freebies}
In addition to active directory, your computer has a local directory of users. These include the Guest user.
Make sure all of the local accounts are disabled.

You can get the local users with: get-LocalUser
And disabling the accounts with: Disable-LocalUser -Name "Guest"

\section{Registry Keys and Autoruns}
<write a program to scan the windows registry and report any anomalies in the keys.
Often keys in things like Firefox or AmazonVM thingie will be creatively named by the red team
so you don't notice them. Need a program to definitively scan all registry keys for malware>

\subsection{What is the Windows Registry?}
Idk

\section{Firewall / Network}

\subsection{Banning IPs}
Banning IPs makes me paranoid of accidentally "red teaming" myself, but it can be a good idea.

I will look into this program called \href{https://github.com/DigitalRuby/IPBan}{IP Ban}.

\section{Red Team Gimmicks}
\epigraph{I was surprised by it, but I wasn’t surprised that I was surprised. I expected a surprise, I just wasn’t sure which one.}
{Fabiano Caruana}

The red team loves gimmicks. It adds some fun and lets them be creative. So expect to be surprised.

\subsection{Visitors}
Assume all visitors are malicious.
If people walk in the room, cover your password sheets. I kept mine in my pocket and only took it out when I needed it. 

\subsection{The Intern}
There is an intern, and they will be polite and unassuming. But their goal is to mess with you! Never take your eyes off the intern!
They can try to plug USBs into your computers, take pictures with your password sheets (don't allow them to take any pictures).
Apparently the intern and her handler will follow any rules you tell them. So tell them to not take picures, install unwanted programs, plug in devices, etc.

\subsection{Other devices in the room}
One time the red team got passwords by having a hidden camera in the room and using it to view passwords on the whiteboard. 
Be careful of physical devices in the room. I think there was speculation that a doll had a camera in it, idk.

\section{Running an Anti Virus}
Windows Defender is good and easy as it comes preinstalled. Of course you should remove any exclusions.

The CCDC rules say programs/information "are completely valid for competition use provided there is no 
fee required to access those resources and access to 
those resources has not been granted based on a previous membership, purchase, or fee"

So we can use free antiviruses, or those with a no-card-required free trial.

Kaspersky seems to be a good antivirus choice.

\href{https://www.sophos.com/en-us/free-tools/hitmanpro}{Hitman Pro}
seems to be a good choice for scanning the file hashes against a large database.
Norton Power Eraser can check for Potentially Unwanted Applications (PUA) and can scan for rootkits.

Kaspersky's TDSSKiller to remove rootkits.

\section{Sysinternals}
The best antivirus is you. It's not empowering, it's mostly just sad.
You need to be able to identify malicious processes using Process Explorer

Great presentation, and he made sysinternals: 
\href{https://youtu.be/vW8eAqZyWeo}{Mark Russinovich on Malware Hunting with Sysinternals}

\href{https://youtu.be/A_TPZxuTzBU}{License to Kill: Malware Hunting with the Sysinternals Tools }

Russinovich also wrote a book (don't pirate it): Windows Sysinternals Administrator’s Reference

Video mentions program modifying the shortcut to use command line arguments. 
Could be a funny way of running malware. Don't trust the shortcut icons!

\subsection{Process Explorer}
Be wary that just because a program is started by System32 does not mean it is safe. <covenant malware>
The red team may give you some low hanging fruit with malware that rings up on VirusTotal.
Suspicious processes include:
conhost.exe / command prompts
powershell prompts
exe files

In process explorer, click "run as task manager." Now Ctrl+Shift+Esc will open Process Explorer (instead of task manager).

\href{https://youtu.be/y2bNLCWHFNs}{Professor K on Process Explorer}

"Put malware to sleep, and only then kill it. 
Then they don't know what's happening.
A lot of malware out there has
the buddy system. Instead of racing against the buddy system
and deleting both malicious files before they respawn each other,
suspend them and then terminate them."

"Also if it is necessary, you don't want to cause irreperable damage to your system."

Things to look for from the Professor K video:
\begin{enumerate}
        \item No "Verified Signer"
        \item VirusTotal scan
        \item Strings of the executable
        \item Process' properties \& the TCP/IP tab
        \item DLLs of the process
        \item Handles of the process
        \item Where the process is launching from
        \item Can view the autostart location from the registry
\end{enumerate}

From Mark Russinovich:
\begin{enumerate}
        \item Purple means a process is packed / encrypted
        \item Have process explorer verify image signatures
        \item Change your refresh rate to 9 seconds so you can see short lived processes
\end{enumerate}

This command checks all executables in the C drive on VirusTotal,
 and opens the browser to the ones that are detected.
\begin{lstlisting}[breaklines=true, columns=fullflexible]
        sigcheck -e -u -vr -s c:\       
\end{lstlisting}

Checks for dlls being signed
\begin{lstlisting}[breaklines=true, columns=fullflexible]
        listdlls -u *       
\end{lstlisting}

\subsection{Autoruns}

\begin{enumerate}
        \item Show only images that are not signed by microsoft
        \item WMI tab autoruns \href{https://medium.com/threatpunter/detecting-removing-wmi-persistence-60ccbb7dff96}{(What is WMI)}
        \item Timestamp column
\end{enumerate}

To remove the progrma that is started by the shell, he changes its registry startp from 
Shell to explorer.exe

\subsection{Process Monitor}
\epigraph{"When in doubt, run process monitor"}{Mark Russinovich}

Useful filter is:
Category isRight only shows changes made to the system


\section{Securing DNS}

\subsection{Disabling Recursive Lookups}
"Disabling recursive lookups can help prevent DNS-based DDoS attacks. 
Recursive DNS queries are when a DNS server processes a domain name 
request on a domain name for which it is not authoritative 
(or has not already cached) by querying the root name servers for 
the IP address of the requested domain name1. A remote attacker could 
spoof a recursive DNS query with a source address of a network they 
wish to cause a denial of service for. The attacker spoofs a query 
with a small payload and causes the DNS server to reply with much 
more data. This floods the target network with answers to questions 
it never asked for2. Disabling open recursion, which causes the server 
to accept DNS 
requests from any IP address, can reduce DNS attack loopholes" -Bing

https://www.cisa.gov/sites/default/files/publications/DNS-recursion033006.pdf

\subsection{DNSSEC}
Domain Name System Security Extensions

Protect against DNS spoofing and cache poisoning

https://www.icann.org/resources/pages/dnssec-what-is-it-why-important-2019-03-05-en

https://www.akadia.com/services/dns\_hardening.html

\section{Securing Active Directory}

One inject mentioned Sean Metcalf.
He runs the site \href{https://adsecurity.org/}{adsecurity.org/}
and I have made a playlist of some useful videos: \href{https://www.youtube.com/playlist?list=PLHkV-wwoQ7s_7vUau-eqiscWoZZNC3EcZ}{playlist}.

\section{Event Logs}


\end{document}