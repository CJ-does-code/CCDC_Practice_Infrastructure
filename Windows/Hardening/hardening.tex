\documentclass{article}
\usepackage{graphicx}
\usepackage{hyperref}
\usepackage{listings}
\usepackage{xcolor}
\usepackage{tikzsymbols}
\usepackage{float}
\usepackage{epigraph}

\lstset{
    basicstyle=\ttfamily,
    backgroundcolor=\color{gray!30},
}

\renewcommand{\epigraphflush}{flushleft}
\title{Hardening}
\begin{document}
\maketitle

\graphicspath{ {./Images/} }
\tableofcontents

\section{Introduction}
\epigraph{I remembered how Albert Camus talks about the concept of resistance. 
The idea is that if you see that you cannot win, do everything in your power to resist. And that memory gave me the determination I needed.}
{Ding Liren}

Assume everything is compromised. Even the laptops you are using for the competition! Be paranoid!

Assume the Red Team has thoroughly reviewed all publicly available material, 
including systems from past years. They've highlighted instances of 
teams reusing passwords over multiple years, and the red team uses a wordlist of 
passwords from previous competitions. One read teamer even claimed the capability to 
enumerate all 12-character passwords in 6 hours (involving renting cloud GPUs). 

Given that the Red Team has 
the entire night to crack hashes, it's important to use good passwords.
I make passwords by thinking of random objects and then scattering some symbols \& numbers around.
The red team's hash-cracking ability is probably especially relevant for windows, 
as kerberos uses hashes extensively. 
%TODO: update with more details about kerberos

\section{Active Directory Freebies}
There are some very quick and easy ways to harden active directory.

\begin{enumerate}
        \item Disable all of the accounts besides the black team and your own
        \item Change your password to something random and good
        \item At some point, you can make a new administrator account with an unusual username. You can keep the old account around as a honeypot if you want
        \item After disabling accounts, you must kick them off of the machine! Merely disabling the account does not kick the user off of the machine
        \item You must also view the ssh connections and kick them off. (both ssh and rdp)
        \item Look at GPO and disable any malicious policies, such as policies that disable microsoft defender.
\end{enumerate}

\section{Local Account Freebies}
In addition to active directory, your computer has a local directory of users. These include the Guest user.
Make sure all of the local accounts are disabled.

You can get the local users with:
\begin{lstlisting}
    get-LocalUser
\end{lstlisting}
And disable the accounts with: 
\begin{lstlisting}
    Disable-LocalUser -Name "Guest"
\end{lstlisting}

\section{Registry Keys and Autoruns}
%TODO do this tryhackme room and explain it
\href{https://tryhackme.com/room/windowsforensics1}{Tryhackme on Windows Forensics}
Zimmerman registry explorer

%TODO: write a program to scan the windows registry and report any anomalies in the keys.
Often keys in things like Firefox or AmazonVM thingie will be creatively named by the red team
so you don't notice them.

\subsection{What is the Windows Registry?}
The Windows Registry is a database storing Windows OS and application settings. 
Watch this \href{https://youtu.be/_U78iAem3uo}{video}

\paragraph{Keys and Subkeys}
These are like folders and subfolders. They can contain other keys (subkeys) and values.

\noindent For instance, HKEY\_LOCAL\_MACHINE\textbackslash SOFTWARE\textbackslash Mozilla might be a key where settings specific to Mozilla software are stored.

\noindent Values:
These are the actual data entries within a key or subkey.
Each value has a \textbf{name, a type, and data.}

\paragraph{Value Types}
\begin{itemize}
\item REG\_SZ: A string value. It's a readable text value and one of the most common types
 you'll encounter.

\item REG\_DWORD: A 32-bit number. Often used for flags, settings, or Boolean values. 
For example, a setting might be 0 (off) or 1 (on).

\item REG\_QWORD: A 64-bit number.

\item REG\_BINARY: Raw binary data.

\item REG\_MULTI\_SZ: Multiple string values, usually in an array. 
Useful when an application needs to store multiple entries in a single value.

\item REG\_EXPAND\_SZ: A string that includes variables to be replaced when called by an application. 
For instance, \%UserProfile\% might expand to the directory path of the current user's profile.
\end{itemize}
There are other types, but these are the most common.

Image File Execution Options (IFEO) in the Windows Registry allow 
developers to specify debugging tools to launch when a particular 
executable starts, but they can be misused by malware to redirect 
or hijack the execution of legitimate programs

\section{Firewall / Network}
%TODO: Learn how the windows firewall & networking works
\subsection{Banning IPs}
Banning IPs makes me paranoid of accidentally ``red teaming" myself, but it can be a good idea.

I will look into this program called \href{https://github.com/DigitalRuby/IPBan}{IP Ban}.

\section{Red Team Gimmicks}
\epigraph{I was surprised by it, but I wasn’t surprised that I was surprised. I expected a surprise, I just wasn’t sure which one.}
{Fabiano Caruana}

The red team loves gimmicks. It adds some fun and lets them be creative. So expect to be surprised.

\subsection{Visitors to the room}
Assume all visitors are malicious.
If people walk in the room, cover your password sheets. I kept mine in my pocket and only took it out when I needed it. 

\subsection{The Intern}
There is an intern, and they will be polite and unassuming. But their goal is to mess with you! Never take your eyes off the intern!
They can try to plug USBs into your computers, take pictures with your password sheets (don't allow them to take any pictures).
Apparently the intern and her handler will follow any rules you tell them. So tell them to not take picures, install unwanted programs, plug in devices, etc.

\subsection{Other devices in the room}
One time the red team got passwords by having a hidden camera in the room and using it to view passwords on the whiteboard. 
Be careful of physical devices in the room. I think there was speculation that a doll had a camera in it, idk.

\section{Running an Anti Virus}
Windows Defender is good and easy as it comes preinstalled. Of course you should remove any exclusions.

\paragraph{Allowed Programs} Rules about The CCDC rules say programs/information ``are completely valid for competition use provided there is no 
fee required to access those resources and access to 
those resources has not been granted based on a previous membership, purchase, or fee"

\noindent So we can use free antiviruses, or those with a no-card-required free trial.

\paragraph{Possible Antiviruses}
\begin{itemize}
\item Kaspersky free
\item \href{https://www.sophos.com/en-us/free-tools/hitmanpro}{Hitman Pro}
seems to be a good choice for scanning the file hashes against a large database.
\item Norton Power Eraser can check for Potentially Unwanted Applications (PUA) and can scan for rootkits.
\item Kaspersky's TDSSKiller to remove rootkits.
\end{itemize}

\section{Sysinternals}
The best antivirus is you. It's not empowering, it's mostly just sad.
You need to be able to identify malicious processes using Sysinternals tools.

\paragraph{Mark Russinovitch} He has a youtube channel with great presentations on sysinternals (and he made sysinternals!): 
\begin{itemize}
\item \href{https://youtu.be/vW8eAqZyWeo}{Mark Russinovich on Malware Hunting with Sysinternals}
\item \href{https://youtu.be/A_TPZxuTzBU}{License to Kill: Malware Hunting with the Sysinternals Tools }
\item Russinovich also wrote a book (don't pirate it): Windows Sysinternals Administrator’s Reference
\end{itemize}

\subsection{Process Explorer}
Process Explorer provides detailed information about processes, including a hierarchical view of parent 
and child processes, as well as in-depth details about resources, handles, and DLLs in use.
It allows you to suspend and kill processes, and can scan the file hashes against VirusTotal.

The red team may give you some low hanging fruit with malware that rings up on VirusTotal.
But if it is not that simple, always remember that suspicious processes include:
conhost.exe / command prompts,
powershell prompts,
and exe files.

\paragraph{Replacing task manager}
In process explorer, click ``run as task manager.'' Now Ctrl+Shift+Esc will open Process Explorer (instead of task manager).

\paragraph{Tips from a Professor K video}
\href{https://youtu.be/y2bNLCWHFNs}{Professor K on Process Explorer}

``Put malware to sleep, and only then kill it. 
Then they don't know what's happening.
A lot of malware out there has
the buddy system. Instead of racing against the buddy system
and deleting both malicious files before they respawn each other,
suspend them and then terminate them."

``Also if it is necessary, you don't want to cause irreperable damage to your system."

Things to look for from the Professor K video:
\begin{enumerate}
        \item No ``Verified Signer"
        \item VirusTotal scan
        \item Strings of the executable
        \item Process' properties \& the TCP/IP tab
        \item DLLs of the process
        \item Handles of the process
        \item Where the process is launching from
        \item Can view the autostart location from the registry
\end{enumerate}

\paragraph{From Mark Russinovich:}
\begin{enumerate}
        \item Purple means a process is packed / encrypted
        \item Have process explorer verify image signatures
        \item Change your refresh rate to 9 seconds so you can see short lived processes
\end{enumerate}

\paragraph{From me}
\begin{enumerate}
    \item Be wary that just because a program is started by System32 does not mean it is safe. 
    I experienced this personally when Eduardo used the Covenant malware against me. 
    It used processes started by System32.
\end{enumerate}    

\paragraph{Sigcheck command}
The sigcheck command with these flags recursively scans the C: 
drive for unsigned executables, verifies their status against VirusTotal.com,
 and opens the browser to the viruses that are detected.
\begin{lstlisting}[breaklines=true, columns=fullflexible]
        sigcheck -e -u -vr -s c:\       
\end{lstlisting}

\paragraph{listdlls}
Lists all processes that have loaded unsigned DLLs
\begin{lstlisting}[breaklines=true, columns=fullflexible]
        listdlls -u *       
\end{lstlisting}

\subsection{Autoruns}

\begin{enumerate}
        \item Show only images that are not signed by microsoft. This will reduce the noise and focus only on the potential malware.
        \item WMI tab autoruns \href{https://medium.com/threatpunter/detecting-removing-wmi-persistence-60ccbb7dff96}{(What is WMI)} WMI-based persistence is less well-known and therefore may be overlooked during manual investigations. 
        \item Timestamp column. The timestamp indicates when a file was created or last modified.
\end{enumerate}

To remove the program that is started by the shell, he changes its registry startup from 
Shell to explorer.exe

\subsection{Process Monitor}
\epigraph{"When in doubt, run process monitor"}{Mark Russinovich}

%TODO: need to watch videos on process monitor and add stuff about it. I think it is significant.

\subsection{TCPView}
TCPView provides a real-time graphical representation of the active TCP and UDP endpoints on a system. 
It displays details about the processes that initiated the connections, 
the local and remote addresses, as well as the state of each connection. 
This tool is particularly useful for network diagnostics, security investigations, 
and understanding the network activity of a system.

You can view active TCP and UDP connections from TCPView, but you must go to
Process Explorer to kill them.

\section{Securing DNS}

\subsection{Disabling Recursive Lookups}
"Disabling recursive lookups can help prevent DNS-based DDoS attacks. 
Recursive DNS queries are when a DNS server processes a domain name 
request on a domain name for which it is not authoritative 
(or has not already cached) by querying the root name servers for 
the IP address of the requested domain name1. A remote attacker could 
spoof a recursive DNS query with a source address of a network they 
wish to cause a denial of service for. The attacker spoofs a query 
with a small payload and causes the DNS server to reply with much 
more data. This floods the target network with answers to questions 
it never asked for2. Disabling open recursion, which causes the server 
to accept DNS 
requests from any IP address, can reduce DNS attack loopholes" -Bing

\href{https://www.cisa.gov/sites/default/files/publications/DNS-recursion033006.pdf}{CISA on DNS recursion}

\subsection{DNSSEC}
Domain Name System Security Extensions

Protect against DNS spoofing and cache poisoning

\paragraph{DNSSEC Resources}
\begin{itemize}
\item \href{https://www.icann.org/resources/pages/dnssec-what-is-it-why-important-2019-03-05-en}{ICANN on DNSSEC}
\item \href{https://www.akadia.com/services/dns_hardening.html}{Akadia DNS Hardening}
\end{itemize}

\section{Securing Active Directory}

\paragraph{Sean Metcalf}
A mobster in the AD space. He runs the site \href{https://adsecurity.org/}{adsecurity.org/}
and I have made a playlist of some useful videos: \href{https://www.youtube.com/playlist?list=PLHkV-wwoQ7s_7vUau-eqiscWoZZNC3EcZ}{playlist}.

\paragraph{Black Hills Information Security Video}
This video is pretty good. \href{https://youtu.be/SdNPUhzYTUc?si=FNYSbLUqm-2Kurxp}{Active Directory Best Practices That Frustrate Pentesters}

\section{Event Logs}
%TODO: Do event logs tryhackme
I am going to do this \href{https://tryhackme.com/room/windowseventlogs}{room} and then get back to you.
But they are definitely very important!

Event Viewer can be launched by typing eventvwr.msc

Event Viewer has three panes.
\begin{enumerate}
\item The pane on the left provides a hierarchical tree listing of the event log providers.
\item The pane in the middle will display a general overview and summary of the events specific to a selected provider.
\item The pane on the right is the actions pane.
\end{enumerate}

\section{Caution with Shortcuts}
Shortcuts can have extra properties and run other programs when you click them.
For example, the Firefox shortcut could really run a script that opens firefox and also some
other malicious program.

\noindent Viewing the properties of a shortcut in the desktop is very straightforward.

\paragraph{Properties of a shortcut in the taskbar}
To view the properties of a shortcut in the taskbar, 
\begin{itemize}
\item Right-click on the shortcut in the taskbar.
\item If the program is already running or pinned to the taskbar, you may see a ``jumplist" (a list of recent items, tasks, or pinned items related to that application). If you see this jumplist:
\item Right-click again on the application's name or icon in that jumplist (i.e., the entry at the top).
\item A context menu will appear. Choose Properties from the context menu.
\end{itemize}

\section{GPO Policies}
Group Policy Objects (GPOs) are a way to centrally manage 
and enforce configuration settings for computers 
and users within an Active Directory domain.

- What are job objects? How can I use them to set limits on program execution / 
% https://learn.microsoft.com/en-us/windows/win32/procthread/job-objects

% https://www.blackhillsinfosec.com/active-directory-best-practices-to-frustrate-attackers-webcast-write-up/
% https://billy-boone.medium.com/group-policy-monitoring-for-cyber-security-59d7953c4da5
\begin{enumerate}
    \item Audit logs for login attempts, file access, system configurations
    \item Account Lockout policy, to prevent brute force attacks. Can I hook it up to an auto-ban as well?
    \item Restrict USB drives (can't hurt. You never know)
    \item Restrict command prompt and powershell
    \item SRP (software restriction policies). Only allows identified applications to run.
    \item Enforce LDAPS. Domain Controller: LDAP server signing requirements
    \item Enforce encryption for SMB, RPC, others?
    \item Setup the firewall from GPO to not have to do it multiple times.
    \begin{itemize}
        \item Block SSH (we only use RDP, can reduce attack surface/scripting ability)
        \item Block WinRM
        \item Allow RDP
        \item Allow LDAP on port 389.
        \item Allow LDAPS on port 636.
        \item Allow DNS on port 53.
        \item Allow Kerberos on port 88.
        \item Allow RDP on port 3389 for specified IP ranges or administrators.
        \item Allow SMB on port 445.
        \item Allow NetBIOS (if required) on ports 137, 138, 139.
        \item Allow NTP (network time protocol) on port 123
        \item 
    \end{itemize}
    \item Task scheduler. Can run antivirus scans on all the computers
    \item Software Installation. Can automatically install antivirus and sysinternals, for example.
    \item Set the local admin passwords through a script given by GPO. Or can use LAPS (Local Administrator Password Solution).
    \item Can send a script to kick off all users who had their accounts disabled
\end{enumerate}

\section{Volatility Framework}
Metasploit and other viruses can run only in RAM. 
We can find them from their network connections, but also from 
RAM forensics.

% https://www.forensicfocus.com/articles/finding-metasploits-meterpreter-traces-with-memory-forensics/

\end{document}