\documentclass{article}
\usepackage{graphicx}
\usepackage{hyperref}
\usepackage{listings}
\usepackage{xcolor}
\usepackage{tikzsymbols}
\usepackage{float}
\usepackage{epigraph}
\usepackage{array}
\usepackage{float}

\renewcommand{\textflush}{flushleft}
\renewcommand{\arraystretch}{1.5}

\lstset{
    basicstyle=\ttfamily,
    backgroundcolor=\color{gray!30},
}
\title{Powershell}
\begin{document}
\maketitle

\graphicspath{ {./Images/} }
\tableofcontents

\section{Introduction}
This file details important powershell commands in one place so you do not have to search to find them.
Note that powershell commands are not case sensitive.

\section{Local Users}
These commands are related to administering the local users (not AD).

\begin{table}[h]
\centering
\begin{tabular}{| m{20em} |c|} % m allows text to wrap around
\hline
Command & Description \\
\hline
Disable-LocalUser -Name "Username" & Disabling a local user \\
\hline
Enable-LocalUser -Name "Username" & Enabling a local user \\
\hline
Get-LocalUser & Viewing local users \\
\hline
New-LocalUser -Name "Username" 
-Password (Read-Host -AsSecureString) & Adding a new local user \\
\hline
\$Password = Read-Host -AsSecureString
Set-LocalUser -Name "Username" -Password \$Password & Changing password of a local user \\
\hline
\end{tabular}
\end{table}

\section{Viewing connections}
Merely disabling an account does not kick the user off, you should manually kick them.
These commands are useful for viewing the current connected users and kicking them off.
\begin{table}[H]
\centering
\begin{tabular}{| m{20em} |c|} % m allows text to wrap around
\hline
Command & Description \\
\hline
query user & View logged-in RDP users \\
\hline
logoff [SessionID] & Disconnecting a user from RDP \\
\hline
Get-NetTCPConnection -State Established | 
Where-Object { \$\_.LocalPort -eq 22 } & View active SSH connections made with OpenSSH \\
\hline
Get-Process -Name sshd & Getting the SSH process \\
\hline
Stop-Process -Id [PID] -Force & Terminating the user's ssh session \\
\hline
\end{tabular}
\end{table}

\section{Installing WSL on Windows Server 2019}
On Server 2022, you can just do wsl --install. But on 2019 it is more complicated. 

Follow these directions: \href{https://learn.microsoft.com/en-us/windows/wsl/install-on-server}{microsoft.com/wsl/install-on-server}

\end{document}